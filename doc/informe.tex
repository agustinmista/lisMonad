\documentclass[a4paper,10pt]{article}
\usepackage[utf8]{inputenc}
\usepackage{fancyhdr, float, graphicx, caption}
\usepackage{amsmath, amssymb}
\usepackage{bm}
\usepackage[margin=1in]{geometry}
\usepackage{multicol}
\usepackage{proof}
\usepackage{titlesec}

\setlength{\inferLineSkip}{4pt}

\titleformat{\subsection}[runin]
  {\normalfont\large\bfseries}{\thesubsection}{1em}{}	
\titleformat{\subsubsection}[runin]
  {\normalfont\normalsize\bfseries}{\thesubsubsection}{1em}{}


\pagestyle{fancy}
\renewcommand{\figurename}{Figura}
\renewcommand\abstractname{\textit{Abstract}}

\fancyhf{}
\fancyhead[LE,RO]{\textit{Intérprete de Cálculo Lambda Simple Tipado}}
\fancyfoot[RE,CO]{\thepage}

%%%%%%%%%%%%%%%%%%%%%%%%%%%%%%%%%%%%%%%%%%%%%%%%%%%%%%%%%%%%

\title{
	%Logo UNR
	\begin{figure}[!h]
		\centering
		\includegraphics[scale=1]{unr.png}
		\label{}
	\end{figure}
	% Pie Logo
	\normalsize
		\textsc{Universidad Nacional de Rosario}\\	
		\textsc{Facultad de Ciencias Exactas, Ingeniería y Agrimensura}\\
		\textit{Licenciatura en Ciencias de la Computación}\\
		\textit{Análisis de Lenguajes de Programación}\\
	% Título
	\vspace{30pt}
	\hrule{}
	\vspace{15pt}
	\Large
		\textbf{Programación Monádica}\\
	\vspace{15pt}
	\hrule{}
	\vspace{30pt}
	% Alumnos/docentes
	\begin{multicols}{2}
	\raggedright
		\large
			\textbf{Alumnos:}\\
		\normalsize
			CRESPO, Lisandro (C-6165/4) \\
			MISTA, Agustín (M-6105/1) \\
			$\;$ \\
			$\;$ \\
	\raggedleft
		\large
			\textbf{Docentes:}\\
		\normalsize
			JASKELIOFF, Mauro\\
			SIMICH, Eugenia\\
			MANZINO, Cecilia\\
			RABASEDAS, Juan Manuel\\
	\end{multicols}
}
%%%%%%%%%%%%%%%%%%%%%%%%%%%%%%%%%%%%%%%%%%%%%%%%%%%%%%%%%%%
\begin{document}
\date{05 de Noviembre de 2015}
\maketitle

\pagebreak
%----------------------------------------------------------
\subsection*{Ejercicio 1.a} 
	\emph{Demostramos que \texttt{State} es una mónada:}
	\\
	\\
	\indent {Probamos que verifique las tres leyes de las mónadas.
	
	Comenzamos probando \texttt{return x $\gg=$ f = f x}}
	\\
	\begin{align*}
  	&\ \ \ \ \ return\ x \gg= f
  	\\ &=\langle \text{ def. \texttt{return} }\rangle
  	\\ &\ \ \ \ \ State\ (\lambda s \rightarrow (v, s')) \gg= f
  	\\ &=\langle \text{ def. $>>=$ }\rangle
  	\\ &\ \ \ \ \ State\ (\lambda s \rightarrow let\ (v,s')=runState\ (State\ (\lambda s \rightarrow (x, s))\ s\ in\ runState\ (f\ v)\ s')
  	\\ &=\langle \text{ def. runState }\rangle
  	\\ &\ \ \ \ \ State\ (\lambda s \rightarrow let\ (v,s')=(\lambda s \rightarrow (x, s))\ s\ in\ runState\ (f\ v)\ s')
  	\\ &=\langle \text{ aplicación }\rangle
  	\\ &\ \ \ \ \ State\ (\lambda s \rightarrow let\ (v,s')=(x, s)\ in\ runState\ (f\ v)\ s')
  	\\ &=\langle \text{ let }\rangle
  	\\ &\ \ \ \ \ State\ (\lambda s \rightarrow runState\ (f\ x)\ s)
  	\\ &=\langle \text{ $\eta$-equivalencia }\rangle
  	\\ &\ \ \ \ \ State\ (runState\ (f\ x))
  	\\ &=\langle \text{ St $\circ$ runState = id }\rangle
  	\\ &\ \ \ \ \ id\ (f\ x)
  	\\ &=\langle \text{ def. id }\rangle
  	\\ &\ \ \ \ \ f\ x
  	\end{align*}
	\\
	\indent {Continuamos probando \texttt{m $\gg=$ return = m}. Lo hacemos por inducción estructural sobre el tipo de datos \texttt{State}.}
	\\
	\begin{align*}
  	&\ \ \ \ \ m \gg= return
  	\\ &=\langle \text{ def. $>>=$ }\rangle
  	\\ &\ \ \ \ \ State\ (\lambda s \rightarrow let\ (v,s')=runState\ m\ s\ in\ runState\ (return\ v)\ s')
  	\\ &=\langle \text{ def. return }\rangle
  	\\ &\ \ \ \ \ State\ (\lambda s \rightarrow let\ (v,s')=runState\ m\ s\ in\ runState\ (State\ (\lambda s \rightarrow (v,s)))\ s')
  	\\ &=\langle \text{ def. runState }\rangle
	\\ &\ \ \ \ \ State\ (\lambda s \rightarrow let\ (v,s')=runState\ m\ s\ in\ (\lambda s \rightarrow (v,s))\ s')
  	\\ &=\langle \text{ aplicación }\rangle
	\\ &\ \ \ \ \ State\ (\lambda s \rightarrow let\ (v,s')=runState\ m\ s\ in\ (v,s'))
  	\\ &=\langle \text{ let }\rangle
	\\ &\ \ \ \ \ State\ (\lambda s \rightarrow runState\ m\ s)
  	\\ &=\langle \text{ aplicación }\rangle
  	\\ &\ \ \ \ \ State\ (runState\ m)
  	\\ &=\langle \text{ St $\circ$ runState = id }\rangle
  	\\ &\ \ \ \ \ id\ m
  	\\ &=\langle \text{ def. id }\rangle
  	\\ &\ \ \ \ \ m
  	\end{align*}
  	\pagebreak
	\\
	\indent {Por último, probamos \texttt{(t $\gg=$ f) $\gg=$ g = t $\gg=$ ($\lambda$x $\rightarrow$ f x $\gg=$ g)}.}
	\\
	\begin{align*}
  	&\ \ \ \ \ ecuacion
  	\\ &=\langle \text{ regla }\rangle
  	\\ &\ \ \ \ \ ecuacion
  	\end{align*}
  	\\
	\begin{align*}
  &\ \ \ \ \ ((St\ h) >>= f) >>= g
  \\ &=\langle \text{ definición $>>=$ }\rangle
  \\ &\ \ \ \ \ St\ (\lambda s \rightarrow let\ (x,s')=h\ s\ in\ runState\ (f\ x)\ s') >>= g
  \\ &=\langle \text{ definición $>>=$ }\rangle
  \\ &\ \ \ \ \ St\ (\lambda t \rightarrow let\ (y, t')=(\lambda s \rightarrow let\ (x,s')=h\ s\ in\ runState\ (f\ x)\ s')\ t\ in\ runState\ (g\ y)\ t')
  \\ &=\langle \text{ aplicación }\rangle
  \\ &\ \ \ \ \ St\ (\lambda t \rightarrow let\ (y, t')=(let\ (x,s')= h\ t\ in\ runState\ (f\ x)\ s')\ in\ runState\ (g\ y)\ t')
  \\ &=\langle \text{ f x :: State s a $\Rightarrow \exists f'\ /\  f\ x = St\ f'$ \label{sec:fxl}}\rangle
  \\ &\ \ \ \ \ St\ (\lambda t \rightarrow let\ (y, t')=(let\ (x,s')= h\ t\ in\ runState\ (St\ f')\ s')\ in\ runState\ (g\ y)\ t')
  \\ &=\langle \text{ runState $\circ$ St = id }\rangle
  \\ &\ \ \ \ \ St\ (\lambda t \rightarrow let\ (y, t')=(let\ (x,s')= h\ t\ in\ f'\ s')\ in\ runState\ (g\ y)\ t')
  \\ &=\langle \text{ propiedad de let \ref{let}}\rangle
  \\ &\ \ \ \ \ St\ (\lambda t \rightarrow let\ (x,s')= h\ t\ in\ (let\ (y, t')=f'\ s'\ in\ runState\ (g\ y)\ t'))
  \\ &=\langle \text{ $\alpha$-equivalencia }\rangle
  \\ &\ \ \ \ \ St\ (\lambda s \rightarrow let\ (x,s')= h\ s\ in\ (let\ (y, t')=f'\ s'\ in\ runState\ (g\ y)\ t'))
  \\ &=\langle \text{ abstracción }\rangle
  \\ &\ \ \ \ \ St\ (\lambda s \rightarrow let\ (x,s')= h\ s\ in\ (\lambda t\rightarrow let\ (y, t')=f'\ t\ in\ runState\ (g\ y)\ t')\ s')
  \\ &=\langle \text{ runState $\circ$ St = id }\rangle
  \\ &\ \ \ \ \ St\ (\lambda s \rightarrow let\ (x,s')= h\ s\ in\ runState (St (\lambda t\rightarrow let\ (y, t')=f'\ t\ in\ runState\ (g\ y)\ t'))\ s')
  \\ &=\langle \text{ definición $>>=$ }\rangle
  \\ &\ \ \ \ \ St\ (\lambda s \rightarrow let\ (x,s')= h\ s\ in\ runState\ (St\ f' >>= g)\ s')
  \\ &=\langle \text{ $St\ f' = f\ x$ \ref{sec:fxl}}\rangle
  \\ &\ \ \ \ \ St\ (\lambda s \rightarrow let\ (x,s')= h\ s\ in\ runState\ (f\ x >>= g)\ s')
  \\ &=\langle \text{ aplicación }\rangle
  \\ &\ \ \ \ \ St\ (\lambda s \rightarrow let\ (x,s')= h\ s\ in\ runState\ ((\lambda x\rightarrow f\ x >>= g)\ x)\ s')
  \\ &=\langle \text{ definición $>>=$ }\rangle
  \\ &\ \ \ \ \ (St\ h) >>= (\lambda x \rightarrow f\ x >>= g)
\end{align*}
	\\
	\pagebreak
%----------------------------------------------------------
\\
\vspace{\fill}
\begin{multicols}{2}
	\hrule
	\vspace{5pt}
	CRESPO, Lisandro \\
	\linebreak
	\hrule
	\vspace{5pt}
	MISTA, Agustín \\
\end{multicols}

\end{document}
